%by ZiMou Feng for the CELE submissions,July 2018.

\documentclass[10pt]{article}
\usepackage[a4paper,left=2.5cm,right=2.5cm,top=2.5cm,bottom=2.5cm]{geometry}
\usepackage{graphicx}
\usepackage{setspace}
\usepackage{fontspec}
\usepackage{array}
\usepackage{natbib}
\defaultfontfeatures{Mapping=text-text}
\setsansfont{Verdana}
\renewcommand*{\familydefault}{\sfdefault}
\linespread{2.5}
\setlength{\parskip}{1\baselineskip}

%I find it surprising that our university would require us to typeset our thesis using the Verdana font. Firstly, Verdana is a sans serif font, and traditionally, body text is typeset in a serif font. Secondly, Verdana is a commercially owned proprietary font, and it seems odd to be that a university would require their students to use any particular commercial product (at least not unless they were supplying it to them for free)

\makeatletter  %to avoid error messages generated by "\@". Makes Latex treat "@" like a letter
\def\class#1{\gdef\@class{#1}}
\def\academicsubject#1{\gdef\@academicsubject{#1}}
\def\submitdate#1{\gdef\@submitdate{#1}}
\def\wordcount#1{\gdef\@wordcount{#1}}

\begin{document}

 \title{Which is the better cancer treatment, traditional chemotherapy  or  nanopharmaceuticals}%your title
 \author{Zimou Feng}%your name
 \class{Kingfisher}%your class
 \submitdate{Monday 30/07/18}%this date is for submission1
 \academicsubject{CELE}
 \wordcount{538} %your word count

\centering{
\includegraphics[width=0.5\columnwidth]{nottingham-logo.png}} 
\par
\vskip 1in 
\par 
   
\begin{tabular}{p{4cm}p{10cm}}
{\bf Name:} & {\@author}\\[50pt]
{\bf Class:} & {\@class}\\[50pt]
{\bf Academic subject:} & {\@academicsubject}\\[50pt]
{\bf The controversy:} & {\@title}\\[50pt]
{\bf Submission date:} & {\@submitdate}\\[50pt]
{\bf Word count:} & {\@wordcount}\\[50pt]
\end{tabular}
\thispagestyle{empty}
\setcounter{page}{0}
\newpage


\begin{flushleft}
Cancer, also known as malignant tumor. When the body is stimulated by various carcinogenic factors, cells of local tissue may lose normal gene regulation and was lead to heterogeneous proliferation and metastasis.Cancer has become the leading cause of global morbidity and mortality according to a report\citep*{torre2015global} by International Agency for Research on Cancer which belongs to World Health Organization.Lacking of effective diagnosis and treatment of cancer has become one of the major problems in contemporary medicine.
\par
Conventional cancer treatment methods mainly include surgery, radiationtherapy and chemotherapy\citep*{peer2007nanocarriers}. Among them, chemotherapeutic drugs are mainly introduced into cancer patients by injection or oral administration to kill cells or inhibit the metastasis and spread of cancer cells. One of the great advantages of chemotherapy is that it is easy to operate and has effective results. However, the clinical anti-tumor drugs have poor targeting, and their concentration in tumor sites is very low, which limits the therapeutic effect of cancer.\citep*{parhi2012nanotechnology,wicki2015nanomedicine}.Due to the poor selectivity of anti-tumor drugs, in addition to the distribution of tumor sites in normal tissues, anti-tumor drugs kill cancer cells and kill normal cells, thus producing strong side effects on cancer patients, Such as hair loss, myelosuppression, cardiotoxicity, nerve and liver toxicity, etc.\citep*{priestman2008some} In addition, many anti-tumor drugs are poorly water-soluble. For example, paclitaxel is hardly soluble in water. Therefore, paclitaxel injections used clinically often use polyoxyethylene castor oil and ethanol for solubilization, but the use of co-solvents during intravenous administration will Cancer patients develop severe allergic reactions\citep*{micha2006abraxane}, such as difficulty breathing, blood pressure drop, angioedema, urticaria, etc.Another major challenge of traditional chemotherapy is the development of drug tolerance during the process, often leading to the ultimate failure of chemotherapy.
\par
In recent years, the rapid development of nanotechnology has brought hope and possibility to open up new ways of cancer treatment.\citep*{cho2008therapeutic,hubbell2012nanomaterials}. Nanomaterials can be used as drug carriers to load drugs by chemical bond coupling, physical solubilization, surface adsorption, etc., thereby significantly increasing the solubility of hydrophobic drugs in water and effectively improving the mode of administration.\citep*{guo2014nanoparticles}. Nanocarriers can bind to tumor cell surface specific antigens or overexpressed receptors by modifying specific tumor targeting ligands on the surface to achieve drug accumulation and accumulation in tumor sites.\citep*{bae2011targeted}. Nanocarriers may enable drugs to controllably release appropriate doses at specific times and locations\citep*{ganta2008review}. However, the efficacy of nanomedicine has not been as expected. Although many nanomedicines have entered the clinical stage or have been approved for marketing, such as Doxil/Caelyx, DaunoXome and Abraxane, these drugs can only improve or improve patients to some extent. The overall survival rate is still unable to achieve the goal of significantly improving patient survival or even completely curing and eliminating tumors.\citep*{chauhan2013strategies}. Another reason for limiting the development of nanomedicine is that it is difficult to synthesize and has a low success rate, resulting in high costs. Moreover, at such a small scale, nanomaterials have unique physicochemical properties different from macroscopic materials, and it is difficult to predict what side effects it will produce.
\par
This controversy is important because of the need of effective treatment of cancer, it is significant to decide whether we should continue to develop traditional chemotherapy or spend money and energy on nanopharmaceuticals.
\end{flushleft}
\newpage

\bibliographystyle{agsm}
\bibliography{bibliography}

\end{document}